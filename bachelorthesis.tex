% Created 2017-11-03 Fri 17:45
% Intended LaTeX compiler: pdflatex
\documentclass[11pt]{article}
\usepackage[utf8]{inputenc}
\usepackage[T1]{fontenc}
\usepackage{fixltx2e}
\usepackage{graphicx}
\usepackage{grffile}
\usepackage{longtable}
\usepackage{wrapfig}
\usepackage{rotating}
\usepackage[normalem]{ulem}
\usepackage{amsmath}
\usepackage{textcomp}
\usepackage{amssymb}
\usepackage{capt-of}
\usepackage[colorlinks=true]{hyperref}
\usepackage{tikz}
\usepackage{minted}
\author{Mario Román}
\date{\today}
\title{Bachelor's thesis\\\medskip
\large Category theory and lambda calculus}
\hypersetup{
 pdfauthor={Mario Román},
 pdftitle={Bachelor's thesis},
 pdfkeywords={},
 pdfsubject={},
 pdfcreator={Emacs 26.0.50 (Org mode 9.0.9)}, 
 pdflang={English}}
\begin{document}

\maketitle
The first part of this thesis describes the untyped and the
simply-typed \textbf{lambda calculus}. The \textbf{Brower-Heyting-Kolmogorov}
\textbf{interpretation} of types leads to a full isomorphism between proofs in
\textbf{Intuitionistic Propositional Logic} and simply-typed lambda calculus
terms. We implement a complete lambda calculus interpreter in the
Haskell programming language that can show the derivation tree in
Gentzen's natural deduction style of any given term.

\quad

The second part describes categories, natural transformations,
adjoints and Kan extensions. \textbf{Lawvere's algebraic theories} are
discussed as a first example of categorical logic and \textbf{topoi} are
studied as an intuitionistic generalization of the Lawvere's
\textbf{Elementary Theory of the Category of Sets}.
This categorical framework is then used to extend the previous
isomorphism between simply-typed lambda calculus and propositional
logic to \textbf{cartesian closed categories}. Robert Seely's 1984 paper
refines this idea to a correspondence between locally closed cartesian
categories and \textbf{Martin-Löf type theories}.

\quad

The thesis finishes with a discussion of the \textbf{Agda} programming language
and how dependent type theories with the \textbf{Voevodsky's Univalence} axiom
can be used to develop an intuitionistic foundation of mathematics.
\end{document}