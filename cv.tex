%%%%%%%%%%%%%%%%%%%%%%%%%%%%%%%%%%%%%%%%%
% Friggeri Resume/CV for A4 paper format
% XeLaTeX Template
% Version 1.0
%
% A4 version author:
% Marvin Frommhold (depressiverobot.com)
% https://github.com/depressiveRobot/friggeri-cv-a4
%
% Original author:
% Adrien Friggeri (adrien@friggeri.net)
% https://github.com/afriggeri/CV
%
% License:
% CC BY-NC-SA 3.0 (http://creativecommons.org/licenses/by-nc-sa/3.0/)
%
% Important notes:
% This template needs to be compiled with XeLaTeX and the bibliography, if used,
% needs to be compiled with biber rather than bibtex.
%
%%%%%%%%%%%%%%%%%%%%%%%%%%%%%%%%%%%%%%%%%

% Options
% 'print': remove colors from this template for printing
% 'nocolor' to disable colors in section headers
\documentclass[print]{friggeri-cv-a4}
\addbibresource{bibliography.bib} % Specify the bibliography file to include publications

\begin{document}
\header{Mario Román}{}{Mathematics and Computer Science student} % Your name and current job title/field

\begin{aside} % In the aside, each new line forces a line break
\section{contact}
Granada,
Spain
~
+0034 693 833838
+0034 958 137645
~
%\href{mailto:mromang08@gmail.com}{mromang08@gmail.com}
%\href{https://m42.github.io}{m42.github.io}
%\href{https://www.linkedin.com/in/mario-roman}{linkedin:mario-roman}
\section{languages}
spanish mother tongue
english fluency (c1)
italian fluency
\section{programming}
Haskell
C++, R
Python, Ruby, Coq
\section{software}
Gnu/Linux
Emacs org-mode
LaTeX
\end{aside}

%----------------------------------------------------------------------------------------
% EDUCATION SECTION
%----------------------------------------------------------------------------------------

\section{education}

\begin{entrylist}

%------------------------------------------------

\entry{2013--2017}{{\normalfont Bachelor degree in} Mathematics}{Universidad de Granada, Spain}
{\emph{Interest in abstract algebra.} Average calification: 9.5/10
}

\entry
{2007--2008}{{\normalfont Bachelor degree in} Computer Science}{Universidad de Granada, Spain}
{\emph{Interest in computation.} Average calification: 9.5/10 \\
Exchange student at \emph{Università degli studi di Milano (2015-2016)}
} 

\entry{2014--2016}{Programming conferences \& courses}{}
{ \emph{Lambda World}, Cádiz, 2016 \\
  \emph{ESSLLI. Logic, Languages and Computation}, Barcelona, 2015 \\
  \emph{Data Science and Big Data}, UNIA, 2014
}

\end{entrylist}


\section{projects}
\subsection{programming projects}

\begin{entrylist}
\entry{2015--Now}{Mikrokosmos}{github.com/M42/mikrokosmos}
{\emph{Hackage: hackage.haskell.org/package/mikrokosmos} \\
  An untyped lambda calculus interpreter written in Haskell
  supporting lazy and eager evaluation.
}
\entry{2014--2015}{GranaSAT Client}{github.com/M42/granasatClient}
{\emph{Git repository: https://github.com/M42/granasatClient} \\
  Software for a satellite student experiment for the European Space Agency BEXUS
  campaign.
}
\end{entrylist}

\subsection{mathematical projects}

\begin{entrylist}
\entry{2016--Now}{Koszul pairs and its applications}{Universidad de Granada, Spain}
{\emph{Git repository: https://github.com/M42/math} \\
  Research grant. Working with the Algebra Department.}
\entry{2015--Now}{LibreIM}{}
{\emph{Web page: http://tux.ugr.es/dgiim/} \\
  Founder of a community of Math\&CS students. I write for a cooperative blog,
  and mantain a Q\&A site. We organize seminars where I have lectured about
  Haskell, Category Theory and Git.
}
\end{entrylist}


\section{awards}

\begin{entrylist}

\entry{2012--2013}{International Mathematical Olympiad (IMO)}{Mar del Plata, Argentina}
{National \textbf{Gold Medal}. \\
International \textbf{Honourable mention}.}

\end{entrylist}

\section{interests}

I love \textbf{abstract algebra}, \textbf{category theory} and its applications to \textbf{functional
programming}. My favorite programming language is \textbf{Haskell}, but I am very interested in
the foundations of programming languages, lambda calculus and dependently typed languages such
as Agda and Coq.

\end{document}