%%%%%%%%%%%%%%%%%%%%%%%%%%%%%%%%%%%%%%%%%
% Friggeri Resume/CV for A4 paper format
% XeLaTeX Template
% Version 1.0
%
% A4 version author:
% Marvin Frommhold (depressiverobot.com)
% https://github.com/depressiveRobot/friggeri-cv-a4
%
% Original author:
% Adrien Friggeri (adrien@friggeri.net)
% https://github.com/afriggeri/CV
%
% License:
% CC BY-NC-SA 3.0 (http://creativecommons.org/licenses/by-nc-sa/3.0/)
%
% Important notes:
% This template needs to be compiled with XeLaTeX and the bibliography, if used,
% needs to be compiled with biber rather than bibtex.
%
%%%%%%%%%%%%%%%%%%%%%%%%%%%%%%%%%%%%%%%%%

% Options
% 'print': remove colors from this template for printing
% 'nocolor' to disable colors in section headers
\documentclass[]{friggeri-cv-a4}
\hypersetup{
     colorlinks   = true,
     citecolor    = gray,
     linkcolor    = green,
     urlcolor     = blue
}  
\addbibresource{bibliography.bib} % Specify the bibliography file to include publications

\begin{document}
\header{Mario Román}{}{Mathematics + Functional Programming} % Your name and current job title/field

\begin{aside} % In the aside, each new line forces a line break
\section{contact}
+34 693 833838
\href{mailto:mromang08@gmail.com}{mromang08@gmail.com}
\href{https://m42.github.io}{m42.github.io}
\href{https://www.linkedin.com/in/mario-roman}{linkedin:mario-roman}
\section{languages}
spanish
english
italian
\section{programming}
Haskell
C++, R
Python, Ruby, Coq
\section{software}
Gnu/Linux
Emacs
LaTeX
\end{aside}



\section{education}

\begin{entrylist}

%------------------------------------------------

\entry{2013--2017}{{\normalfont Bachelor degree in} Mathematics}
{\href{http://www.ugr.es/en/}{University of Granada}, Spain}
{\emph{Emphasis in abstract algebra.} GPA: 9.51/10
}

\entry{2013--2017}{{\normalfont Bachelor degree in} Computer Science}
{\href{http://www.ugr.es/en/}{University of Granada}, Spain}
{\emph{Emphasis in computation.} GPA: 9.47/10 \\
  \href{http://ec.europa.eu/programmes/erasmus-plus/node_en}{Exchange student} at the
  \href{http://www.unimi.it/ENG/}{University of Milan} (2015--2016)
} 

\entry{2016}{{\normalfont Attended} \href{http://www.lambda.world/}{Lambda World} conference}{}{On Functional Programming}

\entry{2015}{{\normalfont Attended} \href{http://parles.upf.edu/llocs/esslli/welcome-esslli-2015}{ESSLLI} conference}{\href{https://www.upf.edu/}{Pompeu Fabra University}, Spain}{On Logic, Languages and Computation}

\entry{2008-2012}{\href{http://thales.cica.es/estalmat/}{Estalmat}}{\href{http://www.unia.es/}{University of Granada, Spain}}{A project to detect and stimulate the precocious mathematical talent.}

% \entry{2014--2016}{Programming conferences \& courses}{}
% { \emph{\href{http://www.lambda.world/}{Lambda World}}, Cádiz, 2016 \\
%   \emph{\href{http://parles.upf.edu/llocs/esslli/welcome-esslli-2015}{ESSLLI. Logic, Languages and Computation}}, Barcelona, 2015 \\
%   \emph{\href{http://sci2s.ugr.es/otherCourses/CienciaDatosBigData}{Data Science and Big Data}}, \href{http://www.unia.es/}{International University of Andalucía}, 2014
% }

\end{entrylist}


%\section{projects}
\subsection{programming projects}

\begin{entrylist}
\entry{2015--Now}{Mikrokosmos}
{\href{https://github.com/M42/mikrokosmos}{github.com/M42/mikrokosmos}}
{\emph{Hackage: \href{https://hackage.haskell.org/package/mikrokosmos}{hackage.haskell.org/package/mikrokosmos}} \\
  An untyped lambda calculus interpreter written in Haskell supporting multiple evaluation strategies.
}
\entry{2014--2015}{GranaSAT Client}
{\href{https://github.com/M42/granasatClient}{github.com/M42/granasatClient}}
{\emph{Git repository: \href{https://github.com/M42/granasatClient}{github.com/M42/granasatClient}} \\
  Software for a satellite student experiment for the
  European Space Agency \href{http://rexusbexus.net/}{BEXUS campaign}.
}
\end{entrylist}

\subsection{mathematical projects}

\begin{entrylist}
\entry{2016--Now}{Koszul pairs and their applications}{University of Granada, Spain}
{\emph{Git repository: \href{https://github.com/M42/math}{github.com/M42/math}} \\
  Research grant. Working with the Algebra Department.}

\entry{2015--Now}{LibreIM}{\href{http://tux.ugr.es/dgiim/}{tux.ugr.es/dgiim/}}
{ Founder and coordinator of a \href{http://tux.ugr.es/dgiim/}{community} of Math\&CS students.
  I am the main contributor to our \href{http://tux.ugr.es/dgiim/blog/}{blog},
  and maintain our \href{http://tux.ugr.es/dgiim/overflow/}{Q\&A site}.
  We organize \href{http://tux.ugr.es/dgiim/awesome/seminars/}{seminars} where I have lectured about
  \href{https://github.com/libreim/introHaskell/blob/master/introHaskell.pdf}{Haskell},
  \href{https://github.com/libreim/introCategorias/blob/master/categorias.pdf}{Category theory}
  and \href{https://github.com/libreim/curryHoward/blob/master/CurryHoward.pdf}{Type theory}.
}
\end{entrylist}


\section{awards}

\begin{entrylist}

\entry{2012--2013}{International Mathematical Olympiad (\href{http://www.imo-official.org/}{IMO})}
{Argentina}
{National \emph{Silver \& Gold Medals} and
\href{https://www.imo-official.org/team_r.aspx?code=ESP&year=2012}{international} \emph{Honourable mention}.}

\end{entrylist}

\section{interests}
I love \textbf{abstract algebra}, \textbf{category theory} and its applications to \textbf{functional
programming}. My favorite programming language is \textbf{Haskell} but I am also very interested in
the foundations of programming languages, lambda calculus and dependently typed languages such
as Coq and Agda.

\end{document}