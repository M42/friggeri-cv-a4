%%%%%%%%%%%%%%%%%%%%%%%%%%%%%%%%%%%%%%%%%
% Friggeri Resume/CV for A4 paper format
% XeLaTeX Template
% Version 1.0
%
% A4 version author:
% Marvin Frommhold (depressiverobot.com)
% https://github.com/depressiveRobot/friggeri-cv-a4
%
% Original author:
% Adrien Friggeri (adrien@friggeri.net)
% https://github.com/afriggeri/CV
%
% License:
% CC BY-NC-SA 3.0 (http://creativecommons.org/licenses/by-nc-sa/3.0/)
%
% Important notes:
% This template needs to be compiled with XeLaTeX and the bibliography, if used,
% needs to be compiled with biber rather than bibtex.
%
%%%%%%%%%%%%%%%%%%%%%%%%%%%%%%%%%%%%%%%%%

% Options
% 'print': remove colors from this template for printing
% 'nocolor' to disable colors in section headers
\documentclass[nocolors]{friggeri-cv-a4}
\hypersetup{
     colorlinks   = true,
     citecolor    = gray,
     linkcolor    = green,
     urlcolor     = blue
}  
\addbibresource{bibliography.bib} % Specify the bibliography file to include publications
\usepackage[english]{babel}
\usepackage{multicol}
\setlength{\multicolsep}{3.0pt plus 1.0pt minus 0.75pt}
\usepackage{enumitem}
\setlist{noitemsep}
\setlist{nosep}

\begin{document}
\header{Mario Román}{}{Mathematics and Computer Science student} % Your name and current job title/field

\begin{aside} % In the aside, each new line forces a line break
  \section{Contact}\\
  Mario Román García\\
+34 693 833838\\
\href{mailto:mromang08@gmail.com}{mromang08@gmail.com}
\href{https://m42.github.io}{m42.github.io}
\href{https://www.linkedin.com/in/mario-roman}{linkedin:mario-roman}
\section{Languages}\\
Spanish\\
English\\
Italian
\section{Programming}
{\small Experience in \textbf{Haskell} and
  the proof assistants \textbf{Coq} and \textbf{Agda};
  object-oriented programming and scripting in \textbf{Ruby}; and imperative
  programming in \textbf{C++}.
}

\section{Software}\\
{\small Technical knowledge of \textbf{Gnu/Linux}.
  Experienced user of \textbf{Emacs} and \textbf{LaTeX}.
}
\end{aside}


\section{Education}

\begin{entrylist}

%------------------------------------------------

\entry{2013-2017}{{\normalfont Bachelor degree in} Mathematics}
{\href{http://www.ugr.es/en/}{University of Granada}, Spain}
{\emph{Emphasis in abstract algebra.} Grade Point Average: 9.55/10.
  {\small
  \begin{multicols}{2}
    \begin{itemize}[topsep=0pt]
    \item Calculus
    \item Geometry, linear algebra
    \item Numerical methods
    \item Probability
    \item Algebra
    \item Analysis and measure theory
    \item Topology
    \item Non-euclidean geometry
    \item Algebraic topology
    \item Galois theory
    \item Mathematical modelling
    \item Statistical inference
    \item Curves and surfaces
    \item Differential equations
    \item Number theory, criptography
    \item Computational algebra
    \item Modern algebra
    \item Logic, discrete mathematics
    \end{itemize}
  \end{multicols}
  }
}

\entry{2013-2017}{{\normalfont Bachelor degree in} Computer Science}
{\href{http://www.ugr.es/en/}{University of Granada}, Spain}
{\emph{Emphasis in computation.} Grade Point Average: 9.35/10
  {\small
    \begin{multicols}{2}
    \begin{itemize}[topsep=0pt]
    \item C++ Programming
    \item System administration
    \item Electronics
    \item Computer architecture
    \item Operative systems
    \item Algorithms
    \item Data structures
    \item Object-oriented programming
    \item Computability theory
    \item Automata and languages
    \item Software engineering
    \item Information theory
    \item Functional programming
    \item Databases
    \item Computer graphics
    \item Artificial intelligence
    \item Metaheuristics
    \item Advanced functional prog.
    \end{itemize}
  \end{multicols}
  }
  
} 

\entry{2015-2017}{Courses and conferences}{}{Attended: {\small
    \begin{itemize}[topsep=0pt]
    \item{\href{https://unimath.github.io/bham2017/}{School on Univalent Mathematics - Birmingham}, on Univalent foundations.}
    \item{\href{https://sites.google.com/view/summerschool2017-eutypes/program}{EUTypes Summer School}, on Homotopy type theory, Agda and Coq.}
    \item{\href{https://sites.google.com/unizar.es/affine-group-schemes-seminar/inicio}{Seminar on Affine group schemes}, Hopf algebras and algebroids.}
    \item{\href{http://parles.upf.edu/llocs/esslli/welcome-esslli-2015}{ESSLLI-Barcelona}, on Logic, Languages and Computation.}
    \item{\href{http://www.lambda.world/}{Lambda World}, on functional programming.}
      \item{\href{http://www.ugr.es/~orientamat/}{OrientaMat - LaTeX course}, volunteering as \textbf{teaching assistant}.}
    \end{itemize}
  }
}

\entry{2008-2012}{\href{http://thales.cica.es/estalmat/}{Estalmat}}{\href{http://www.unia.es/}{University of Granada, Spain}}{A project to detect and stimulate the precocious mathematical talent.}

% \entry{2014--2016}{Programming conferences \& courses}{}
% { \emph{\href{http://www.lambda.world/}{Lambda World}}, Cádiz, 2016 \\
%   \emph{\href{http://parles.upf.edu/llocs/esslli/welcome-esslli-2015}{ESSLLI. Logic, Languages and Computation}}, Barcelona, 2015 \\
%   \emph{\href{http://sci2s.ugr.es/otherCourses/CienciaDatosBigData}{Data Science and Big Data}}, \href{http://www.unia.es/}{International University of Andalucía}, 2014
% }

\end{entrylist}




\subsection{Mathematical projects}

\begin{entrylist}
  \entry{2017-2018}{Category theory and λ-calculus}{Bachelor's thesis (\href{https://github.com/M42/lambda.notes}{in progress})}
  { {\small \textit{Advisor: \href{https://scholar.google.es/citations?user=gvq9UmMAAAAJ&hl=en&oi=ao}{Prof. Pedro A. García-Sánchez}.}} \\
    Bachelor thesis on the relationship between type theories and categorical logic.
    Martin-Löf type theories are regarded as the internal language of locally closed
    cartesian categories and presented as a foundation of
    mathematics. Agda and Coq are used to prove theorems in Homotopy type theory.
   }
  
\entry{2016-2017}{Koszul pairs and their applications}{\href{http://github.com/M42/math}{unpublished}}
{ Research grant. Working with the Algebra Department on Homology theory from
  a categorical perspective.}}


\end{entrylist}

% \section{projects}
\newpage
\subsection{Computer science projects}



\begin{entrylist}
\entry{2016-Now}{Mikrokosmos}
{\href{https://github.com/M42/mikrokosmos}{github.com/M42/mikrokosmos}}
{\emph{Hackage: \href{https://hackage.haskell.org/package/mikrokosmos}{hackage.haskell.org/package/mikrokosmos}} \\
  An didactic free software λ-calculus interpreter written in Haskell supporting multiple evaluation strategies and exemplifying the Curry-Howard isomorphism.
}
\entry{2014-2015}{GranaSAT Client}
{\href{https://github.com/M42/granasatClient}{github.com/M42/granasatClient}}
{\emph{Git repository: \href{https://github.com/M42/granasatClient}{github.com/M42/granasatClient}} \\
  Software for a satellite student experiment for the
  European Space Agency \href{http://rexusbexus.net/}{BEXUS campaign}.
}
\end{entrylist}

\textit{A complete portfolio can be found at \href{https://m42.github.io}{m42.github.io}.}

\section{Publications}

\begin{entrylist}
  \entry{2016}{A comparison of implementations of basic evolutionary algorithm operations in different languages}{}
  {
    DOI: 10.1109/CEC.2016.7743980 \\
    Conference: \href{http://ieeexplore.ieee.org/document/7743980/2016}{IEEE Congress on Evolutionary Computation (CEC)}
    }
\end{entrylist}

\section{Awards \& Grants}

\begin{entrylist}

  \entry{2017-2018}{Collaboration Grant}{Algebra department, University of Granada}
  {By virtue of which I can develop my \textbf{bachelor's thesis}, I administer the deparment
    servers and I develop \textbf{didactic material} and assist in the teaching of the course
    \textbf{"Logic and Programming"}.
  }

  \entry{2015-2016}{\href{http://ec.europa.eu/programmes/erasmus-plus/node_en}{Erasmus+ Grant}}{\href{http://www.unimi.it/ENG/}{University of Milan}}{
    Exchange student at the
    \href{http://www.unimi.it/ENG/}{University of Milan} for a year. Studying at the
    \href{http://www.di.unimi.it/ecm/home/didattica/international-studies/activity/courses}{department of computer science}.
  }
  
\entry{2012--2013}{International Mathematical Olympiad (\href{http://www.imo-official.org/}{IMO})}
{Argentina}
{National \emph{Gold \& Silver Medals} and
  \href{https://www.imo-official.org/team_r.aspx?code=ESP&year=2012}{international} \emph{Honourable mention}.}


\end{entrylist}

\section{Interests}

I am passionate about \textbf{logic}, \textbf{abstract algebra}, \textbf{category theory} and their
applications to \textbf{functional programming}. Since I program with dependently
typed languages such as Coq and Agda, I have become increasingly more interested
in type-theoretical foundations of mathematics, categorical
logic and \textbf{topos theory}.

I am actively involved in the \textbf{divulgation of mathematics} and computer
science at a universitary level. I weekly organize mathematics and
computer-science talks at my university; where I have been able to
develop teaching skills and a deeper understanding of mathematics.

\begin{entrylist}
\entry{2014--Now}{LibreIM}{\href{http://libreim.github.io/}{libreim.github.io/}}
{ Founder and coordinator of a \href{http://libreim.github.io/}{community} of Math\&CS students.
  I am the main contributor to our \href{http://libreim.github.io/blog/}{blog} and
  the organizer of weekly \href{http://libreim.github.io/dgiim/awesome/seminars/}{seminars} where I have lectured about
  \href{https://github.com/libreim/haskell}{Haskell},
  \href{https://github.com/libreim/introCategorias/blob/master/categorias.pdf}{Category theory}
  and \href{https://github.com/libreim/curryHoward/blob/master/CurryHoward.pdf}{Type theory},
  among other topics.
}

\entry{2002-2012}{School of Music}{Granada, Spain}{ Completed
  Foundational and Professional levels. Specialization in \textit{violin},
  \textit{music theory} and \textit{composition}.  }
\end{entrylist}

\end{document}