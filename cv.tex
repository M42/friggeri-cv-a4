%%%%%%%%%%%%%%%%%%%%%%%%%%%%%%%%%%%%%%%%%
% Friggeri Resume/CV for A4 paper format
% XeLaTeX Template
% Version 1.0
%
% A4 version author:
% Marvin Frommhold (depressiverobot.com)
% https://github.com/depressiveRobot/friggeri-cv-a4
%
% Original author:
% Adrien Friggeri (adrien@friggeri.net)
% https://github.com/afriggeri/CV
%
% License:
% CC BY-NC-SA 3.0 (http://creativecommons.org/licenses/by-nc-sa/3.0/)
%
% Important notes:
% This template needs to be compiled with XeLaTeX and the bibliography, if used,
% needs to be compiled with biber rather than bibtex.
%
%%%%%%%%%%%%%%%%%%%%%%%%%%%%%%%%%%%%%%%%%

% Options
% 'print': remove colors from this template for printing
% 'nocolor' to disable colors in section headers
\documentclass[nocolors]{friggeri-cv-a4}
\hypersetup{
     colorlinks   = true,
     citecolor    = gray,
     linkcolor    = green,
     urlcolor     = blue
}  
\addbibresource{bibliography.bib} % Specify the bibliography file to include publications
\usepackage[english]{babel}
\usepackage{multicol}
\setlength{\multicolsep}{3.0pt plus 1.0pt minus 0.75pt}
\usepackage{enumitem}
\setlist{noitemsep}
\setlist{nosep}

\begin{document}
\header{Mario Román}{}{PhD student in Computer Science, Tallinn University of Technology} % Your name and current job title/field

\begin{aside} % In the aside, each new line forces a line break
  \section{Contact}\\
  Mario Román García\\
+34 693 833838\\
\href{mailto:mromang08@gmail.com}{mromang08@gmail.com}
\href{https://mroman42.github.io}{mroman42.github.io}
\href{https://www.linkedin.com/in/mario-roman}{linkedin:mario-roman}
\section{References}\\
\href{}{Pawel Sobocinski}\\
\textit{Tallinn University of Technology}\\
\href{mailto:pawel.sobocinski@ttu.ee}{pawel.sobocinski@ttu.ee }\\
\quad
\href{https://www.mansfield.ox.ac.uk/dr-carmen-constantin}{Carmen Constantin}\\
\textit{University of Oxford}\\
\href{mailto:i.m.carmen@gmail.com}{i.m.carmen@gmail.com}\\
\quad\\
\href{https://scholar.google.com/citations?user=gvq9UmMAAAAJ&hl=en&oi=ao}{Pedro García-Sánchez}\\
\textit{University of Granada}\\
\href{mailto:pedro@ugr.es}{pedro@ugr.es}\\
\quad\\
% \href{https://scholar.google.com/citations?user=HULIk-QAAAAJ&hl=en&oi=sra}{Francisco Herrera}\\
% \textit{University of Granada}\\
% \href{mailto:herrera@decsai.ugr.es}{herrera@decsai.ugr.es}\\
\quad\\



\section{Languages}\\
\textit{Spanish (native)\\
English (C2) \\
Italian (fluent)}
\section{Programming}
{\small Links to projects \begin{itemize}
  \item \href{https://github.com/mroman42/mikrokosmos}{Haskell}
  \item \href{https://github.com/mroman42/granasatClient}{C}
  \item \href{https://github.com/mroman42/programming}{C++}
    \item \href{https://github.com/mroman42/cosmoi-emacs}{Emacs Lisp}
    \item \href{https://github.com/mroman42/ctlc}{Agda}
    \item \href{https://github.com/UniMath/UniMath/commit/e26dacaec4bd1e0b26616fcecc9bba5e507cb941}{Coq}
    \item \href{https://github.com/homalg-project/FinSetsForCAP/pull/24}{GAP}
    \item \href{https://github.com/mroman42/funzionale}{F#}
    \item \href{https://github.com/mroman42/Napakalaki}{Java}
    \item \href{https://github.com/mroman42/linguaggi/blob/master/scala/brainfuck.scala}{Scala}
    \item \href{https://github.com/mroman42/rbNapakalaki}{Ruby}
    \end{itemize}
}

\section{Software}\\
{\small \begin{itemize}
  \item \textbf{Gnu/Linux}
  \item \textbf{Emacs}
  \item \textbf{LaTeX}
  \item \textbf{Git/GitHub/GitLab}
  %\item \textbf{MySQL}
  \end{itemize}
}
\end{aside}


 \section{Publications}

 \begin{entrylist}
   \entry{2019}{Profunctor optics, a categorical update}{}
   {
     \textit{Clarke, Elkins, Gibbons, Loregian, Milewski, Pillmore, Román.} \\
     We describe new bidirectional data accessors in terms of the Tambara modules studied by Pastro and Street. \\
     In preparation. Submitted to \href{https://lics.siglog.org/lics20/}{LiCS'20}. \\
     }

   \entry{2019}{Discokitty, an implementation of the DisCoCat framework}{}
   {
     \textit{Mario Román.} \\
     In progress. \\
     }

   \entry{2018}{Mikrokosmos: an educational lambda calculus interpreter (\href{https://jose.theoj.org/papers/10.21105/jose.00029}{pdf})}{}
   {
     \textit{Mario Román} \\
     DOI: \href{https://jose.theoj.org/papers/10.21105/jose.00029}{10.21105/jose.00029} \\
     The Journal of Open Source Education.
   }
   
   \entry{2016}{A comparison of implementations of basic evolutionary algorithm operations in different languages}{}
   {
     \textit{J.J. Merelo-Guervós, I. Blancas, P. Castillo, G. Romero, V. Rivas, M. García-Valdez, A. Hernández-Aguila, M. Román.} \\
     DOI: 10.1109/CEC.2016.7743980 \\
     Conference: \href{http://ieeexplore.ieee.org/document/7743980/2016}{IEEE Congress on Evolutionary Computation (CEC)}
   }

\end{entrylist}



\section{Education}

\begin{entrylist}

%------------------------------------------------

\entry{2019-ongoing}{{\normalfont \textbf{PhD Computer Science} }{}}
  {} {\href{https://www.ox.ac.uk/admissions/graduate/courses/msc-mathematics-and-foundations-computer-science?wssl=1}
    {\normalsize{Tallinn University of Technology}}, Supervisor: \href{https://www.ioc.ee/~pawel/}{Pawel Sobocinski}. \\ \emph{First year. Applied category theory and profunctor optics.}} 

\entry{2018-2019}{{\normalfont \textbf{MSc. Mathematics} }{\normalfont \textbf{and Foundations of Computer Science}}}
  {}
  {\href{https://www.ox.ac.uk/admissions/graduate/courses/msc-mathematics-and-foundations-computer-science?wssl=1}
    {\normalsize{\textbf{University of Oxford}}}, (Grade average: \textbf{91.4/100}) \\ \emph{\href{https://github.com/mroman42/optics-mfocs}{MSc. Dissertation on Profunctor optics}. Supervisor: \href{https://www.cs.ox.ac.uk/people/jeremy.gibbons/}{Jeremy Gibbons}} {
    %   \scriptsize
    % \begin{multicols}{2}
    %   \begin{itemize}[topsep=0pt]
    %   \item Categories, Proofs (95/100), Constantin
    %   \item Homological Algebra (85/100), Henriques
    %   \item Quantum Comp. Sci. (81/100), Coecke
    %   \item Category Theory (91/100), Kirwan
    %   \item Categorical Quantum (95/100), Vicary
    %   \item Distributional Models (90/100), Coecke
    %   \end{itemize}
    % \end{multicols}
  }}
  
  
\entry{2012-2018}{Bachelor degree in Mathematics}
{\href{http://www.ugr.es/en/}{University of Granada}}
{\emph{Emphasis in abstract algebra.} Grade Point Average: 9.55/10.
  { 
    % \scriptsize
  % \begin{multicols}{2}
  %   \begin{itemize}[topsep=0pt]
  %   \item Calculus
  %   \item Geometry, linear algebra
  %   \item Numerical methods
  %   \item Probability
  %   \item Algebra
  %   \item Analysis and measure theory
  %   \item Topology
  %   \item Non-euclidean geometry
  %   \item Algebraic topology
  %   \item Galois theory
  %   \item Mathematical modelling
  %   \item Statistical inference
  %   \item Curves and surfaces
  %   \item Differential equations
  %   \item Number theory, criptography
  %   \item Computational algebra
  %   \item Modern algebra
  %   \item Logic, discrete mathematics
  %   \end{itemize}
  % \end{multicols}
  }
}

\entry{2012-2018}{Bachelor degree in Computer Science}
{\href{http://www.ugr.es/en/}{University of Granada}}
{\emph{Emphasis in computation.} Grade Point Average: 9.35/10 \\ \\
  { \textbf{Bachelor's thesis. Category theory and lambda calculus.} \\
    \emph{Awarded Best Mathematics-related Bachelor's thesis (300€), University of Granada (2017-18).
    Advisor: \href{https://scholar.google.es/citations?user=gvq9UmMAAAAJ&hl=en&oi=ao}{Pedro A. García-Sánchez}.}
    % \scriptsize
  %   \begin{multicols}{2}
  %   \begin{itemize}[topsep=0pt]
  %   \item C++ Programming
  %   \item System administration
  %   \item Computer architecture
  %   \item Operative systems
  %   \item Algorithms
  %   \item Data structures
  %   \item Object-oriented programming
  %   \item Computability theory
  %   \item Automata and languages
  %   \item Software engineering
  %   \item Information theory
  %   \item Functional programming
  %   \item Databases
  %   \item Computer graphics
  %   \item Artificial intelligence
  %   \item Metaheuristics
  %   \end{itemize}
  % \end{multicols}
  }
  
} 

% \entry{2006-2012}{High School}
% {\href{http://www.juntadeandalucia.es/averroes/centros-tic/18700037/helvia/sitio/}{IES Soto de Rojas}}
% {\emph{Emphasis on Science.} Grade Point Average: 10/10 \\
%   { \emph{Music School}, specialty on Violin, 2002-12.
%   } 
% }

\entry{2015-ongoing}{Courses and conferences}{}{Attended: {\small
    \begin{itemize}[topsep=0pt]
    \item{\href{http://events.cs.bham.ac.uk/syco/2/}{SYCO 2 (Strathclyde)} and \href{http://events.cs.bham.ac.uk/syco/3/}{SYCO 3 (Oxford)}, on applied category theory.}
    \item{\href{https://homalg-project.github.io/capdays-2018/program/}{CAP Days - Siegen}, on computable homological algebra and categories.}
    \item{\href{https://unimath.github.io/bham2017/}{School on Univalent Mathematics 2017} and \href{https://unimath.github.io/bham2019/}{2019} - Univ. Birmingham.}
    \item{\href{https://sites.google.com/view/summerschool2017-eutypes/program}{EUTypes Summer School}, on Homotopy type theory, Agda and Coq.}
    \item{\href{https://sites.google.com/unizar.es/affine-group-schemes-seminar/inicio}{Seminar on Affine group schemes}, Hopf algebras and algebroids.}   %\item{\href{http://parles.upf.edu/llocs/esslli/welcome-esslli-2015}{ESSLLI-Barcelona}, on Logic, Languages and Computation.}
    %\item{\href{http://www.lambda.world/}{Lambda World}, on functional programming.}
      \item{\href{http://www.ugr.es/~orientamat/}{LaTeX course - OrientaMat}, teaching assistant.}
    \end{itemize}
  }
}

\entry{2008-2012}{\href{http://thales.cica.es/estalmat/}{Estalmat}}{\href{http://www.unia.es/}{University of Granada, Spain}}{A project to detect and stimulate the precocious mathematical talent.}

% \entry{2014--2016}{Programming conferences \& courses}{}
% { \emph{\href{http://www.lambda.world/}{Lambda World}}, Cádiz, 2016 \\
%   \emph{\href{http://parles.upf.edu/llocs/esslli/welcome-esslli-2015}{ESSLLI. Logic, Languages and Computation}}, Barcelona, 2015 \\
%   \emph{\href{http://sci2s.ugr.es/otherCourses/CienciaDatosBigData}{Data Science and Big Data}}, \href{http://www.unia.es/}{International University of Andalucía}, 2014
% }

\end{entrylist}



\newpage
\subsection{Predoctoral research experience}

\begin{entrylist}
  \entry{2019-ongoing}{Applied Category Theory School 2019}{\href{https://www.cs.ox.ac.uk/ACT2019/}{ACT 2019}}
  { Accepted into the project on
    \textit{Traversal optics and profunctors}, directed by \textit{\href{https://bartoszmilewski.com/2019/01/05/act2019-school-call-for-participation/}{Bartosz Milewski}}.
  }
  
  \entry{2017-2018}{Collaboration and research Fellowship, department of
    Algebra}{}
  { {\small \textit{Advisor: \href{https://scholar.google.es/citations?user=gvq9UmMAAAAJ&hl=en&oi=ao}{Prof. Pedro A. García-Sánchez}.}} \\
    During my Bachelor's thesis. On the categorical semantics of
    lambda calculus. Studying Martin-Löf type theory as the internal
    language of locally closed cartesian categories. Agda is used to
    provide examples of formalized univalent mathematics.  }

   


\end{entrylist}


% \section{projects}


\subsection{Computer science projects}



\begin{entrylist}
\entry{2016-2018}{Mikrokosmos}
{\href{https://github.com/mroman42/mikrokosmos}{github.com/mroman42/mikrokosmos}}
{\emph{Hackage: \href{https://hackage.haskell.org/package/mikrokosmos}{hackage.haskell.org/package/mikrokosmos}} \\
  An didactic free software λ-calculus interpreter written in Haskell supporting multiple evaluation strategies and exemplifying the Curry-Howard isomorphism.
}
\entry{2014-2015}{GranaSAT Client}
{\href{https://github.com/mroman42/granasatClient}{github.com/mroman42/granasatClient}}
{\emph{Git repository: \href{https://github.com/mroman42/granasatClient}{github.com/mroman42/granasatClient}} \\
  Software for a satellite student experiment for the
  European Space Agency \href{http://rexusbexus.net/}{BEXUS campaign}.
}
\end{entrylist}

\textit{A complete portfolio can be found on
  \href{https://github.com/mroman42}{GitHub}.}
I have made contributions to
\begin{itemize}
\item The \href{https://github.com/UniMath/UniMath}{\textbf{UniMath} Coq library}, by Voevodsky, Greyson, Ahrens et al.
\item \href{https://homalg-project.github.io/CAP_project/CAP/}{\textbf{Categories, Algebra and Programming} GAP package},
  by Gutsche, Posur et al.
\end{itemize}

 
 
\section{Awards and Grants}

\begin{entrylist}

  \entry{2017-2018}{Collaboration Grant (2000€)}{}{Algebra department, I administered the deparment servers, developed
    \textit{didactic material} and the \textit{\href{https://github.com/mroman42/mikrokosmos}{Mikrokosmos}}
    interpreter, and assisted in the teaching of the course
    \textbf{"Logic and Programming"} \textit{(\href{http://grados.ugr.es/informatica/pages/infoacademica/guias_docentes/201516/cuarto/ingenieriadelsoftware/complementos/logicayprogramacion/!}{Lógica y programación})}.}

  \entry{2015-2016}{\href{http://ec.europa.eu/programmes/erasmus-plus/node_en}{Erasmus+ Grant (5442€)}}{\href{http://www.unimi.it/ENG/}{University of Milan}}{
    Exchange student at the
    \href{http://www.unimi.it/ENG/}{University of Milan} for a year. Studying at the
    \href{http://www.di.unimi.it/ecm/home/didattica/international-studies/activity/courses}{department of computer science}.
  }
  
\entry{2012--2013}{International Mathematical Olympiad, \href{http://www.imo-official.org/}{IMO} (\textasciitilde{} 1100€)}
{Argentina}
{National \emph{Gold (2012) and Silver (2011) Medals} and an
  \href{https://www.imo-official.org/team_r.aspx?code=ESP&year=2012}{International} \emph{Honourable mention} (2012).}


\end{entrylist}

\section{Interests}
I am also actively involved in the
\textbf{divulgation of mathematics} and computer science at a
universitary level.

\begin{entrylist}
  \entry{2014--2018}{LibreIM, a local community for Maths\&CS}{\href{http://libreim.github.io/}{libreim.github.io/}}
  { \vspace{-2ex}\begin{itemize}
    \item Founded a local \href{http://libreim.github.io/}{community} of Math\&CS students in Granada.
      \item Organized seminars, gathering interested students and speakers.
      \item Main contributor to our \href{http://libreim.github.io/blog/}{blog}.
      \item Prepared seminars on \href{https://github.com/mroman42/libreim-constructiva/blob/master/constructiva.pdf}{constructive mathematics (PDF)}. 
        \href{https://github.com/libreim/haskell}{Haskell} and \href{https://github.com/libreim/introCategorias/blob/master/categorias.pdf}{Category
          theory} and , among others.
      \item Worked with a group of students to keep the LibreIM organization working without me after leaving Granada.
    \end{itemize}
  }
\end{entrylist}
\end{document}
